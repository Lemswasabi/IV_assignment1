\documentclass{article}
\usepackage[utf8]{inputenc}
\usepackage{xcolor}

\title{Assignment 1}
\author{tba}
\date{\today}

\begin{document}

\maketitle


\section{WHAT?}

\textcolor{gray}{Critique and re-design of an existing graphic}

\section{WHY?}

\textcolor{gray}{To deal with a given static graphic, to identify deficiencies
of the visual representation, and to find a more suitable way to present the
data at hand.}

\section{HOW?}

\textcolor{gray}{
\begin{enumerate}
		\item Each group is given an individual graphic to critique and re-design.
		\item In a first step you should argue why the given presentation is not
			appropriate to represent the data in an efficient way. Please argue about
			the deficiencies of the graphic with respect to the design principles we
			covered in the lecture (Data-Ink-Ratio, Visual Clutter, ...).  Hand in a
			short text about your argumentation.
		\item In a second step you should re-design the graphic (i.e., find a better
			suited visual representation for the data at hand). Numerical values that
			cannot be read from the original graphic should be estimated. You can use
			any tool to create the re-designed graphic (or scan hand-drawn designs),
			you may find tools like Tableau, or programming libraries like D3 are
			useful in this context (the use of one of these tools is not mandatory,
			however). Hand in the re-designed graphic. If numbers cannot be derived
			from the given graphic they should be estimated If the original graphic
			contains more than 4 charts, choose 3-4 of these charts and relate them in
			your re-design, if applicable
		\item In a third step you should argue why you chose to represent the data
			that way. What did you change? Why is this representation better suited
			than the original one? Hand in a short text about your argumentation.
    \item Your submission should not exceed 5 pages in length.
\end{enumerate}}

\end{document}
