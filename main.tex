\documentclass{article}
\usepackage[utf8]{inputenc}
\usepackage{xcolor}

\title{Assignment 1}
\author{tba}
\date{\today}

\begin{document}

\maketitle


\section{WHAT?}

\textcolor{gray}{Critique and re-design of an existing graphic}

\section{WHY?}

\textcolor{gray}{To deal with a given static graphic, to identify deficiencies
of the visual representation, and to find a more suitable way to present thehttps://www.overleaf.com/project/5ea6c3016fe2d90001bd596f
data at hand.}

\section{HOW?}

\textcolor{gray}{
\begin{enumerate}
		\item Each group is given an individual graphic to critique and re-design.
		\item In a first step you should argue why the given presentation is not
			appropriate to represent the data in an efficient way. Please argue about
			the deficiencies of the graphic with respect to the design principles we
			covered in the lecture (Data-Ink-Ratio, Visual Clutter, ...).  Hand in a
			short text about your argumentation.
		\item In a second step you should re-design the graphic (i.e., find a better
			suited visual representation for the data at hand). Numerical values that
			cannot be read from the original graphic should be estimated. You can use
			any tool to create the re-designed graphic (or scan hand-drawn designs),
			you may find tools like Tableau, or programming libraries like D3 are
			useful in this context (the use of one of these tools is not mandatory,
			however). Hand in the re-designed graphic. If numbers cannot be derived
			from the given graphic they should be estimated If the original graphic
			contains more than 4 charts, choose 3-4 of these charts and relate them in
			your re-design, if applicable
		\item In a third step you should argue why you chose to represent the data
			that way. What did you change? Why is this representation better suited
			than the original one? Hand in a short text about your argumentation.
    \item Your submission should not exceed 5 pages in length.
\end{enumerate}}

\section{Analysis and criticism}
\subsection{Summary of this graphic}

    \begin{itemize}
       \item Nutritional comparison between processed and natural/wholesome food
       \item Comparisons:
       \begin{itemize}
         \item Cost per calorie
         \item Calories per 100g
         \item Sugar per 100g
         \item A pack of chips is more energy dense than compared to an apple
         \item Wholesome food is more expensive, the ratio of value to calorie for an apple is higher as for a bag of potato chips
       \end{itemize}
       \item Suggestions of the infographic:
       \begin{itemize}
           \item Stick to the periphery of the supermarket to find the natural food
           \item Processed foog containing the most calories by weight are located at the center of the grocery store
       \end{itemize}
       \item 
    \end{itemize}

\subsection{Criticism}
In this section, the three sub-infographics will be criticised according to the information design cues from the lecture notes.

Some notes:
\begin{itemize}
    \item The author uses the \textit{Bottom up} approach which is data driven
    \item The graphics is a conventional representation an used the controlled visualization paradigm (reason: data driven)
    \item Which requires the attentive perception of the reader/viewer
    \item Attentive approach is slow to perceive and easy to forget information
    \item Based on visual processing paradigm and requires the following from the viewer:
    \begin{itemize}
        \item Parallel processing to extract low level properties of the visual scene
        \item Pattern perception
    \end{itemize}
    \item Deficiency for viewers with color blindness, see produce aisle/section which is represented in red and green
    \item Graphic in 3 dimensions which makes it way more complex than it is
    \item The food graphical components obscure the actual information and data \rightarrow hard to perceive actual infographic
    \item Preattentive processin:
    \begin{itemize}
        \item No immediate understanding
        \item No preattentive attributes except for food labels, but many are obscured by other components
        \item Hard for immediate perception
        \item 
    \end{itemize}
    \item The perspective suggests the entry of the store in the graphic is on the right side, and that the higher numbers of the table are closer in perspective \rightarrow based on the western left to right reading perspective it should be the other way around
    \item The differences between the three tables are difficult to compare due to too much information and bright color
    \item Data-Users-Tasks triangle:
    \begin{itemize}
        \item Expressiveness - Are the visual elements expressive enough? \rightarrow Even too much
        \item Effectiveness - Is the information displayed effectively? \rightarrow Not at all
        \item Appropriateness - Is the visualization appropriate? \rightarrow Yes, it's perfect for the subject
    \end{itemize}
    \item 
    \item 
\end{itemize}
\subsubsection{First graphic}
\subsubsection{Second graphic}
\subsubsection{Third graphic}


\section{Redesign}
\subsection{Suggestions}

\section{Assessment}

\end{document}
