\section{Analysis and criticism}

\subsection{Analysis of the infographic}

% corrected
The infographic \textit{Nutritional Values} by Dan Mariglio in
Figure~\ref{assignmentGraphic} pictures the nutritional comparison between
processed and natural/wholesome food. The author visualizes three different
infographics for the different food categories; the first graph compares the
cost per calorie, another one contrasts the calories per \textit{100g} and the
last graphic shows how much sugar is contained in \textit{100g}. The author
gives the example that wholesome food is more expensive since the ratio of value
to calorie for an apple is higher compared to a bag of potato chips. Mariglio
based his visualizations on the layout of common supermarkets and suggests the
reader sticking to the periphery of the supermarket to find the natural food.
Processed food containing the most calories by weight is located at the centre
of the grocery store.

\subsection{Criticism of the visual design}

% corrected
In this section, the three infographics in Figure~\ref{assignmentGraphic} will
be criticised according to the visual design principles from the lecture notes.
These design principles were defined by Edward Tufte\cite{Tufte2001}. Since the
three illustrations are very similar in design, the criticism will cover the
overall design principle choices.

\subsubsection{Perception and cognition}

% corrected
On first sight of the infographics as a viewer, there is no immediate
understanding of the given information. There are too many bright colours which
can distract the viewer to localize the main message or information of the
visual design. Further, the visual design carries no preattentive attributes,
except the black food labels but many of them are obscured by other visual
components or are drowning in the variety of colours which are more distracting
than communication the information with clarity.  Thus, on first sight, the
infographics are flawed for immediate perception.

% corrected
The reasons which cause the infographics to be unsatisfactory are the poorly
chosen design principles by the author. Instead of the \textit{preattentive
processing} paradigm, the author uses the data-driven, Bottom-up approach.
Therefore, the graphs should be perceived with the visual processing paradigm
and require from the viewer an attentive perception and parallel processing to
extract low-level properties of the visual scene. The attentive approach is slow
to perceive and viewers easily forget the main message and essential information
of the infographic. Furthermore, the visual representation is not friendly for
viewers who might suffer from colour blindness since the author uses a wide
subset of colours in his visual design.

\subsubsection{Deficiency of design priniciples by Edward Tufte}

% correct
\textbf{Principles of Graphical Excellence.} Complex ideas should be
communicated with clarity, precision and efficiency. Further, The viewer should
be able to grasp the ideas of the visualization quick with the least amount of
ink in the smallest space. The infographics by Dan
Mariglio~\ref{assignmentGraphic} are inefficient and do not incorporate these
principles to guarantee clarity.

% corrected
\textbf{Principles of Graphical Integrity.} The principles of graphical
integrity define that a graphical design should be detailed and labelled with
clarity. However, the food graphical components in Dan Mariglio's
infographics~\ref{assignmentGraphic} obscure the actual information and data
found in the background. This deficiency impedes the viewer to perceive the
message communicated by the author. The Lie factor of this infographic does not
provide the desired proportionality of represented numbers since it includes
redundant design variation instead of showing data variation. Furthermore, the
graphics are represented in 3 dimensions which makes it more complex than the
data it visualizes. According to Tufte, the visual representation's dimensions
should not exceed the dimensions of the data. 

\textbf{Principles of Data Graphics.}

Tufte's first principle of data graphics defines, \textit{'Above all else show
the data'}~\cite{Tufte2001}. Although, Mariglio was determined to illustrate the
data redundantly in form of visual food components. The author was too focused
on communicating the message where the wholesome food is located in a
supermarket and neglected the more important data.

The graphical ink should favorably present new information and eliminate
redundant and non-data ink without any loss of the data, thus maximizing the
\textit{data-ink ratio}

% The elegance of visual design, defined by reached state where no elements cannot be removed and not the other way around of nothing to add
% Deemphasize non-data ink


% Nothing stands out since data-ink and non-data ink are given the same weight
% The viewers' eyes are drawn to contrast found in the 3d components which leads the reader from the actual data
% Different colours even though the same dataset leads the reader to think the difference it dataset
% Chartjunk 


% 5. Maximize data density, within reason (Data density = number of entries/area of data graphic) 

\textbf{Rules by Stephen Few about using colour in charts and graphs.}

% Use different colours only when they correspond to differences of meaning
% in the data 
% here for example for the vegetables multiple colours are used
% although that's not necessary

% To guarantee that most people who are colourblind can distinguish groups of data
% that are colour-coded, avoid using a combination of red and green in the same
% display.

% Use soft natural colours to display most information and bright and/or dark
% colours to highlight information that requires greater attention
