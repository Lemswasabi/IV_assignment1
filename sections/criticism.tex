\section{Analysis and criticism}

\subsection{Analysis of the infographic}

% corrected
The infographic \textit{Nutritional Values} by Dan
Marsiglio~\cite{DanMarsiglio2008} pictures the nutritional comparison between
processed and wholesome food. The author visualizes three different infographics
for the different food groups and compares their nutritional values; the first
graph compares the cost per calorie, the following one contrasts the calories
per \textit{100g} and the last graphic shows how much sugar is contained in
\textit{100g}. The author gives the example that wholesome food is more
expensive since the ratio of value to calorie for an apple is higher compared to
a bag of potato chips. Marsiglio based his visualizations on the layout of a
common supermarket and suggests the reader sticking to the perimeter of the
supermarket to find the natural food.  Processed food containing the most
calories by weight is located at the centre of the grocery store.

\subsection{Criticism of the visual design}

% corrected
In this section, the three infographics \textit{Nutritional Values} by Dan
Marsiglio~\cite{DanMarsiglio2008} will be criticised according to the visual
design principles from the lecture notes. These design principles were defined
by Edward Tufte\cite{Tufte2001}. Since the three illustrations are very similar
in design, the criticism will cover the overall design principle choices.

\subsubsection{Perception and cognition}

% corrected
On first sight of the infographics as a viewer, there is no immediate
understanding of the given information. There are too many bright colours which
can distract the viewer to localize the main message or information of the
visual design. Further, the visual design carries no preattentive attributes,
except the black food labels but many of them are obscured by other visual
components or are drowning in the variety of colours which are more distracting
than communicating the information with clarity. Thus, on first sight, the
infographics are flawed for immediate perception.

% corrected
The reasons which cause the infographics to be unsatisfactory are the poorly
chosen design principles by the author. Instead of the \textit{preattentive
processing} paradigm, the author uses the data-driven, Bottom-up approach.
Therefore, the graphs should be perceived with the visual processing paradigm
and require from the viewer an attentive perception and parallel processing to
extract low-level properties of the visual scene. The attentive approach is slow
to perceive and viewers easily forget the main message and essential information
of the infographic. Furthermore, the visual representation is not friendly for
viewers who might suffer from colour blindness since the author uses a wide
subset of colours in his visual design.

\subsubsection{Deficiency of design priniciples by Edward Tufte}

% correct
\textbf{Principles of Graphical Excellence.} Complex ideas should be
communicated with clarity, precision and efficiency. Further, The viewer should
be able to grasp the ideas of the visualization quick with the least amount of
ink in the smallest space. The infographics by Dan
Marsiglio~\cite{DanMarsiglio2008} are inefficient and do not incorporate these
principles to guarantee clarity.

% corrected
\textbf{Principles of Graphical Integrity.} The principles of graphical
integrity define graphical designs to be detailed and labelled with clarity.
However, the food visual components in Dan Marsiglio's
infographics~\cite{DanMarsiglio2008} obscure the actual information and data
found in the background. This deficiency impedes the viewer to perceive the
message communicated by the author. The \textit{Lie factor} of this infographic
does not provide the desired proportionality of represented numbers since it
includes redundant design variation instead of showing data variation.
Furthermore, the graphics are represented in three dimensions which makes it
more complex than the data it visualizes. According to Tufte, the visual
representation's dimensions should not exceed the dimensions of the data. 

\textbf{Principles of Data Graphics.}

% corrected
Tufte's first principle of data graphics defines, \textit{'Above all else show
the data'}~\cite{Tufte2001}. Although, Marsiglio was determined to illustrate the
data redundantly in the form of visual food components. The author was too
focused on communicating the message where the wholesome food is located in a
supermarket and neglected the more important data.

% corrected
The graphical ink should favourably present new information. It should eliminate
or deemphasize redundant and non-data-ink without any loss of data, thus
maximizing the \textit{data-ink ratio} and \textit{data density}. Whereas, in
Marsiglio's infographic, the viewer is overwhelmed with redundant information in
the form of \textit{3d} visual food components and unnecessary food labels which
reduces the \textit{data-ink ratio}. The graph in the background which
represents the concrete data is cluttered with \textit{chartjunk}.

% corrected
The data of the infographic does not stand out since the non-data-ink is given
more weight than the data-ink. Further, the viewers' eyes are drawn to contrast
found in the \textit{3d} components which lead their attention away from the
data. Different colours are used for the different food choices even though they
are represented in a single related dataset. This could lead the viewer to
differentiate the relationship of the data entries.

% corrected
\textbf{Rules by Stephen Few about using colour in charts and
graphs.}~\cite{Stephen2004} Different colours should only be used when the data
entries in the dataset should be distinguished. However, with Marsiglio's
infographic the different colours were used only for aesthetic purposes and to
complement the visual food components. The differentiation of colours is misused
since the data entries are related in the dataset. Furthermore, the author uses
a combination of red and green in the same illustration which can be
inconvenient for viewers with colour blindness to distinguish visual elements.
To highlight important information, bright and dark colours should be used.
However, the infographic highlights the non-essential components and distracts
the viewer from the data in the background which are displayed in light colours.
