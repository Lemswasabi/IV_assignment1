\section{Assessment of redesign}

% corrected
In this section, the redesign of Dan Marsiglio's
infographic~\cite{DanMarsiglio2008} will be described according to Edward
Tufte's design principles~\cite{Tufte2001}.\\

% corrected
\textbf{De-emphasize the non-data-ink.} The original graphic has a very low
data-ink ratio. The visual representation of the groceries in the graphics is
redundant since the infographic is already labelled with the correct food
groups. About 60\% of the graphics are non-data-ink, which causes a very low
data-ink ratio.
% I would say 80-90% ?

% needs rephrasing
\textbf{De-emphasize the realistic layout of the grocery store.} We
think, that because of the message of this infographic is important. the
purpose is to show, that being stuck to the perimeter of the supermarket
makes it easier to buy healthy products. We kept this property, but our
graphic shows the layout from above in two dimensions. For the floor-plan is
used only a light grey colour, which is not misleading for the reader.

% corrected
\textbf{Augmented data-ink.} In the graphic of Dan Marsiglio, the reader has to
read all the values explicitly from the three-dimensional graphic, which is
inefficient. In our graphic, the values are directly labelled. It guarantees the
reader easier and faster reading and interpretation.

% needs rephrasing
\textbf{Preattentive processing.} As in the last point mentioned, the
perspectivic projection of the three dimensional space makes the reading of
the original infographic very hard. At first sight it's also ambiguous,
what is the purpose of the changing height of the columns. Our solution
has a two dimensional floor-plan with normalized graphs, which makes easy to
compare the three attribution (cost, calories, sugar) of the diverse groups of
groceries.

% corrected
\textbf{Declutter.} A significant part of the original infographic is chartjunk
as mentioned above. We removed many non-essential elements of the graph which
includes the three-dimensional perspective. 

% corrected
\textbf{Colors.} We only used the colour blue and a couple of its shades to make
it more readable for viewers with colour blindness. Dark shades for processed
food and light for the wholesome food.\\

% corrected
\textbf{Regrouping.} The data in the original chart is split into three separate
figures representing the values of cost, calories and sugar, respectively.
Splitting up the visual representation of the information for one type of
product into three different charts makes the overall comparison of the products
hard. This is the reason we regrouped the data. Our infographic has a separate
diagram for each product group, where each graph is uniformly scaled for better
comparison, corresponding to Tufte's principle of graphical integrity.

% corrected
\textbf{Background-foreground.} In Dan Marsiglio's graphic the chartjunk takes
up a lot of space in the foreground, and the data seems to be only secondary, it
is represented in the background with light colours and thin lines. Our graphic
uses saturated colours for the data in the foreground. Whereas, light and neutral
grey colour were used for the supermarket floor-plan in the background to not
overshadow the highlighted data.\\

The above mentioned design suggestions are summarized in the visual graphic in
Figure~\ref{redesignGraphic} in the Appendix.
