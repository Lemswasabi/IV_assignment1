\section{Assessment of redesign}

How we decided to redesign the infographic:

\begin{itemize}
	\item \textbf{Deemphasize the non data-ink:} The original graphic has a very low data-ink ratio. Visual representation of the groceries in the graphics is totally unnecessary, since they are labeled with text too. About 60\% of the graphics are non-data ink, which causes a very low data-ink ratio 
    \item \textbf{Deemphasize the realistic layout of the grocery store:} We think, that because of the message of this infographic is important: the purpose is to show, that being stuck to the periphery of the supermarket makes it easier to buy healthy products. We kept this property, but our graphic shows the layout from above in 2 dimensions. For the floor-plan is used only a light grey color, which is not misleading for the reader.
	\item \textbf{Augmented data-ink:} In the graphic of Dan Marsiglio the reader has to read all the values explicitly from the 3 dimensional graphic, which is inefficient. In our graphic the values are directly labeled. It guarantees the reader easier and faster reading and interpretation.
    \item \textbf{Preattentive processing:} As in the last point mentioned, the perspectivic projection of the 3 dimensional space makes the reading of the original infographic very hard. At first sight it's also ambiguous, what is the purpose of the changing height of the columns.  
	\item declutter
	\item regroup the data
	
	% \item Push non-data ink into the background (supermarket blueprint)
	% \item Display data in the foreground
	
	Oh yeah, we used the colour blue to be user friendly to colour blind viewers
\end{itemize}



% link to appendix of new design


