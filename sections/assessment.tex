\section{Assessment of redesign}

In this section, the redesign of Dan Marsiglio's~\cite{DanMarsiglio2008} infographic will be described according to Edward Tuftes's~\cite{Tufte2001} design principles.

\begin{itemize}
	\item \textbf{Deemphasize the non data-ink:} The original graphic has a very
		low data-ink ratio. Visual representation of the groceries in the graphics
		is totally unnecessary, since they are labeled with text too. About 60\% of
		the graphics are non-data ink, which causes a very low data-ink ratio

	\item \textbf{Deemphasize the realistic layout of the grocery store:} We
		think, that because of the message of this infographic is important: the
		purpose is to show, that being stuck to the periphery of the supermarket
		makes it easier to buy healthy products. We kept this property, but our
		graphic shows the layout from above in 2 dimensions. For the floor-plan is
		used only a light grey color, which is not misleading for the reader.
		
	\item \textbf{Augmented data-ink:} In the graphic of Dan Marsiglio the reader
		has to read all the values explicitly from the 3 dimensional graphic, which
		is inefficient. In our graphic the values are directly labeled. It
		guarantees the reader easier and faster reading and interpretation.

	\item \textbf{Preattentive processing:} As in the last point mentioned, the
		perspectivic projection of the 3 dimensional space makes the reading of
		the original infographic very hard. At first sight it's also ambiguous,
		what is the purpose of the changing height of the columns. Our solution
		has a 2 dimensional floor-plan with normalized graphs, which makes easy to
		compare the 3 attribution (cost, calories, sugar) of the diverse groups of
		groceries.

	\item \textbf{Declutter:} A significant part of the original infographic is
		chartjunk (like already discussed in the first point). We threw away every
		non-essential elements of the graph, including the 3 dimensional
		perspective. We used only blue color to make it easy to read for color-blind
		people as well, and two shades of it: dark for the processed food, and light
		for the non-processed

	\item \textbf{Regrouping:} The data in the original chart is split up to 3
		separate figures representing the values of the cost, calories and sugar,
		respectively. Splitting up the visual representing of the information for
		one type of product into 3 different charts makes the overall comparison of
		the products hard. That is the reason, why we've regrouped the data: our
		infographic has a separate  diagram for each productgroup, with every data
		normalized between 0 and 1, corresponding Tufte's principle of graphical
		integrity. This makes for the reader easy to compare the properties of the
		productgroups.

	\item \textbf{Background-foreground:} in Dan Marsiglio's graphic the chartjunk
		takes up a lot of space in the foreground, and the data seems to be only
		secondary, it is represented in the background with light colors and thin
		lines. Our graphic uses saturated colors for the data in the foreground, and
		light, neutral grey color for the supermarket floor-plan (the secondary
		data) in the background.
		
\end{itemize}

The above mentioned design suggestions are summarized in the visual graphic in Figure~\ref{redesignGraphic}.
