\section{Assessment of redesign}

% corrected
In this section, the redesign of Dan Marsiglio's
infographic~\cite{DanMarsiglio2008} will be described according to Edward
Tufte's design principles~\cite{Tufte2001}.\\

% corrected
\textbf{De-emphasize the non-data-ink.} The original graphic has a very low
data-ink ratio. The visual representation of the groceries in the graphics is
redundant since the infographic is already labelled with the correct food
groups. About 80\% of the graphics are non-data-ink, which causes a very low
data-ink ratio.
% I would say 80-90% ?

% corrected
\textbf{De-emphasize the realistic layout of the grocery store.} The goal is to
preserve the author's suggestion to stick to the perimeter of the supermarket
for healthy food. However, a light grey colour was used instead to illustrate
the two dimensional floor-plan of the grocery store without distracting the
viewer much from the data.

% corrected
\textbf{Augmented data-ink.} In the graphic of Dan Marsiglio, the reader has to
read all the values explicitly from the three-dimensional graphic, which is
inefficient. In our graphic, the values are directly labelled. It guarantees the
reader easier and faster reading and interpretation.

% corrected
\textbf{Preattentive processing.} The infographic carries no preattentive
attributes apart from the black food labels which got lost in a wide range of
colours. The colours were misused and mostly distracting. In the redesigned
infographic, we only used a few shades of blue. Most of the graphs which
represent the processed food were given less weight with the light shades of
blue. Whereas the wholesome food was given darker shades to be highlighted for
the viewer.

% corrected
\textbf{Declutter.} A significant part of the original infographic is chartjunk
as mentioned above. We removed many non-essential elements of the graph which
includes the three-dimensional perspective. 

% corrected
\textbf{Colors.} We only used the colour blue and a couple of its shades to make
it more readable for viewers with colour blindness. Light shades for processed
food and dark for the wholesome food. We deliberatly decided to remove the
food-coded colours and used colour only to relate the different graph values
between the food groups.\\

% corrected
\textbf{Regrouping.} The data in the original chart is split into three separate
figures representing the values of cost, calories and sugar, respectively.
Splitting up the visual representation of the information for one type of
product into three different charts makes the overall comparison of the products
hard. This is the reason we regrouped the data. Our infographic has a separate
diagram for each product group, where each graph is uniformly scaled for better
comparison, corresponding to Tufte's principle of graphical integrity.

% corrected
\textbf{Background-foreground.} In Dan Marsiglio's graphic the chartjunk takes
up a lot of space in the foreground, and the data seems to be only secondary, it
is represented in the background with light colours and thin lines. Our graphic
uses saturated colours for the data in the foreground. Whereas, a light grey
colour was used for the supermarket floor-plan in the background to not
overshadow the highlighted data.\\

The above mentioned design suggestions are summarized in the visual graphic in
Figure~\ref{redesignGraphic} in the Appendix.
